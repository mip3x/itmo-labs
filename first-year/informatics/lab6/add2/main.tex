\documentclass{book} 
\usepackage[letterpaper,top=2cm,bottom=2cm,left=0.5cm,right=1cm,marginparwidth=1.75cm]{geometry}
\usepackage{multirow}
\usepackage[utf8x]{inputenc}
\usepackage[english,russian]{babel}
\usepackage{cmap}
\usepackage{multicol}
\usepackage[margin=1in]{geometry}
\usepackage{graphicx} 
\usepackage{fancyhdr}
\usepackage{hyperref}
\usepackage[dvipsnames]{xcolor}
\usepackage{graphicx}
\usepackage{tikz}
\usepackage{geometry}
\usepackage{caption}
\usepackage{titlesec}
\usepackage[dotinlabels]{titletoc}
\usepackage{amsmath, amsfonts, amssymb, amsthm, mathtools, wasysym}
\usepackage{tkz-euclide}
\usepackage{gensymb}
\usepackage[absolute,overlay]{textpos}

\newenvironment{forceright}
  {\par\rightskip=0pt plus 1fil\relax}
  {\par}

\hypersetup{
    colorlinks=true,
    linkcolor=blue,
    filecolor=magenta,      
    urlcolor=cyan,
    pdftitle={Overleaf Example},
    pdfpagemode=FullScreen,
    }

\renewcommand{\headrulewidth}{0pt}

\pagestyle{fancy}
\fancyhf{}
\fancyhead[C]{96 КНИГА III ПРЕДЛ. IV. ТЕОРЕМА}
\renewcommand{\headrulewidth}{0pt}
\fancyfoot[C]{\thepage}


\begin{document}

\begin{multicols}{2}[\columnsep=0cm] 
\noindent

\begin{tikzpicture}
    \fill[yellow!40!orange] (0,-1.7) -- (0.7,-1.95) arc(0:13:1.5) -- cycle;
    
    \fill[MidnightBlue!110] (0,-1.8) -- (0.6,-1.7) arc(0:85:0.5) -- cycle;

    \draw[dashed, line width=2pt] (0, -1.8) -- (0.3, 0.1)  node[anchor=south, scale=0.8]{$F$};

    \draw[line width=2pt, draw=red] (-3.2,-\radius + 2.42) -- (3.91,-1);

    \draw[line width=2pt, draw=black] (-3.90,-1) -- (\radius + 3.2,-\radius + 2.42);

    \def\radius{\dimexpr0.2\textwidth\relax}
    \draw[line width=2.5pt, draw=MidnightBlue!110] (0,0) circle (\radius);

    \node[above, scale=0.8] at (-4.2,-1.1) {$A$};
    \node[above, scale=0.8] at (-3.4, -\radius + 33) {$B$};
    \node[above, scale=0.8] at (\radius - 18, -\radius + 35) {$C$};
    \node[above, scale=0.8] at (4.2,-1.3) {$D$};
    \node[above, scale=0.8] at (0,-2.4) {$E$};
    
\end{tikzpicture}

\columnbreak
  
\noindent

\setlength{\columnsep}{-3cm}
\begin{multicols}{2}

\includegraphics[width=80pt, height=80pt]{assets/e.png}

\columnbreak

\noindent
\textit{сли в круге две прямые, не проходящие через центр, пересекаются, они не делят друг друга пополам.}

\end{multicols}

Если одна из прямых проходит через центр, очевидно, она ее не может рассекать пополам другая прямая, не проходящая через центр.

Но если ни одна из прямых 
\begin{tikzpicture}
    \draw[line width=2pt, black] (0, 0) -- (1, 0) 
    node[right, above, pos=1, scale=0.5] {C} 
    node[left, above, pos=0, scale=0.5] {A};
\end{tikzpicture}
или
\begin{tikzpicture}
\draw[line width=2pt, red!60!black] (0, 0) -- (1, 0) 
    node[right, above, pos=1, scale=0.5, black] {D}
    node[left, above, pos=0, scale=0.5, black] {B};
\end{tikzpicture}
не проходит через центр, проведем
\begin{tikzpicture}
\draw[dashed, line width=2pt, black] (0, 0) -- (1, 0) 
    node[right, above, pos=1, scale=0.5] {F} 
    node[left, above, pos=0, scale=0.5] {E};
\end{tikzpicture}
из центра к точке их пересечения.

\begin{align*}
    &\text{Если}
    \begin{tikzpicture}
        \draw[line width=2pt, black] (0, 0) -- (1, 0) 
        node[right, above, pos=1, scale=0.5] {C} 
        node[left, above, pos=0, scale=0.5] {A};
    \end{tikzpicture}
    \text{делится пополам,}& \\
    &\begin{tikzpicture}
        \draw[dashed, line width=2pt, black] (0, 0) -- (1, 0) 
        node[right, above, pos=1, scale=0.5] {F} 
        node[left, above, pos=0, scale=0.5] {E};
    \end{tikzpicture} 
    \perp \text{ей (пр. \(\text{пр.III}._3\))}& 
\end{align*}

\begin{center}
\(\therefore\)
\begin{tikzpicture}
    \fill[yellow!40!orange] (0,-1.9) -- (0.6780,-2.1) arc(0:20:1.6) -- cycle;

    \fill[MidnightBlue!110] (0,-1.9) -- (0.6,-1.7) arc(0:85:0.5) -- cycle;

    \node[above, scale=0.5] at (0.6780,-2.4) {C};
    \node[above, scale=0.5] at (0,-2.3) {E};
    \node[above, scale=0.5] at (0.3, -1.1) {F};
\end{tikzpicture}
\(=\)
\includegraphics[width=20pt, height=20pt]{assets/angle.png} \\
и если 
\begin{tikzpicture}
\draw[line width=2pt, red!60!black] (0, 0) -- (1, 0) 
    node[right, above, pos=1, scale=0.5, black] {D} 
    node[left, above, pos=0, scale=0.5, black] {B};
\end{tikzpicture}
делится пополам, \\
\begin{tikzpicture}
        \draw[dashed, line width=2pt, black] (0, 0) -- (1, 0) 
        node[right, above, pos=1, scale=0.5] {F} 
        node[left, above, pos=0, scale=0.5] {E};
    \end{tikzpicture} \perp 
    \begin{tikzpicture}
\draw[line width=2pt, red!60!black] (0, 0) -- (1, 0) 
    node[right, above, pos=1, scale=0.5, black] {D} 
    node[left, above, pos=0, scale=0.5, black] {B};
\end{tikzpicture}
ей (пр. \(\text{пр.III}._3\))\\

\(\therefore\)
\begin{tikzpicture}
    \fill[MidnightBlue!110] (0,-1.9) -- (0.6,-1.7) arc(0:85:0.5) -- cycle;

    \node[above, scale=0.5] at (0.6780,-2) {D};
    \node[above, scale=0.5] at (0,-2.14) {E};
    \node[above, scale=0.5] at (0.3, -1.1) {F};
\end{tikzpicture} \(=\)
\includegraphics[width=20pt, height=20pt]{assets/angle.png} ;\\
и \(\therefore\) \begin{tikzpicture}
    \fill[MidnightBlue!110] (0,-1.9) -- (0.6,-1.7) arc(0:85:0.5) -- cycle;

    \node[above, scale=0.5] at (0.6780,-2) {D};
    \node[above, scale=0.5] at (0,-2.14) {E};
    \node[above, scale=0.5] at (0.3, -1.1) {F};
\end{tikzpicture} \(=\)
\begin{tikzpicture}
    \fill[yellow!40!orange] (0,-1.9) -- (0.6780,-2.1) arc(0:20:1.6) -- cycle;

    \fill[MidnightBlue!110] (0,-1.9) -- (0.6,-1.7) arc(0:85:0.5) -- cycle;

    \node[above, scale=0.5] at (0.6780,-2.4) {C};
    \node[above, scale=0.5] at (0,-2.3) {E};
    \node[above, scale=0.5] at (0.3, -1.1) {F};
\end{tikzpicture}; \\
часть равна целому, что невозможно.\\

\(\therefore\)
\begin{tikzpicture}
        \draw[line width=2pt, black] (0, 0) -- (1, 0) 
        node[right, above, pos=1, scale=0.5] {C} 
        node[left, above, pos=0, scale=0.5] {A};
    \end{tikzpicture}
    \text{и}&
    &\begin{tikzpicture}
        \draw[line width=2pt, red!60!black] (0, 0) -- (1, 0) 
    node[right, above, pos=1, scale=0.5, black] {D} 
    node[left, above, pos=0, scale=0.5, black] {B};
    \end{tikzpicture} 

    не делят друг друга пополам. \\
\end{center}

\flushright{ч.т.д.}

\end{multicols}

\end{document}